\documentclass[english]{article}
\usepackage[T1]{fontenc}
\usepackage[latin9]{inputenc}
\usepackage{geometry}
\geometry{verbose,tmargin=2cm,bmargin=2cm,lmargin=2cm,rmargin=2cm}
\setlength{\parskip}{\medskipamount}
\setlength{\parindent}{0pt}
\usepackage{bm}
\usepackage{amsmath}
\usepackage{babel}
\begin{document}

\title{Australian Equine Influenza model}

\maketitle

\section{infer-ausei version 4.0}

The following documentation describes the model used in infer-ausei
version 4.0. Please refer to the \texttt{README} file for details
of how to make inference on this model.


\subsection{Between-property model}

The state of properties during the epidemic is modelled as a continuous
time SIO (Susceptible, Infected, Onset) process, with pairwise transmission
rate
\[
\beta_{ij}(t)=\mu\cdot h_{i}(t-I_{i})n_{i}a_{i}^{\xi}\cdot n_{j}a_{j}^{\zeta}\theta^{\bm{1}[j\in\mathcal{V}(t)]}\frac{\delta}{\delta^{2}+\rho_{ij}^{2}}\hspace{1em}i\in\mathcal{I},j\in\mathcal{S}
\]
and assumes a fixed 2 day time from infection to onset \emph{at the
between property level}. $\mu$ is the baseline transmission rate,
$n_{k}$ is the number of horses on property $k$, $a_{k}$ is the
area of property $k$ with non-linear parameters $\xi$ and $\zeta$
giving the effect (note that if these non-linear parameters were 0,
this would imply no effect of area). $\theta$ is the effect of vaccination
on the susceptible properties with $\mathcal{V}(t)$ the set of vaccinated
properties at time $t$. $\rho_{ij}$ is the Euclidean distances (map-units/1000)
between centroids(?) of the properties $i$ and $j$, and $\delta$
is the decay parameter for our Cauchy distance kernel. The function
$h_{i}(t-I_{i})$ is the infectivity of property $i$, taken to be
proportional to the number of animals infected according to the within-property
model described below.

At time $t$, the \emph{infectious pressure} $\lambda_{j}(t)$ on
property $j$ is
\[
\lambda_{j}(t)=\epsilon_{t}+\sum_{i\in}\beta_{ij}(t)
\]
where
\[
\epsilon_{t}=\begin{cases}
\epsilon_{0} & \mbox{ if }t<10\\
\epsilon_{10} & \mbox{if }10\leq t<44\\
\epsilon_{44} & \mbox{if }44\leq t
\end{cases}
\]
corresponding to control events at $t=10$ and $t=44$ (remind me
what these events were?).


\subsection{Within-property model}

At the \emph{within property level}, $h_{i}(s)=I_{i}(s)/n_{i}$ function
returns the proportion of horses infected on property $i$ at time
$s$ after $i$'s infection time, where $I_{i}(s)$ is the solution
to a deterministic SEIR model 

\begin{eqnarray*}
\frac{dS_{i}(t)}{dt} & = & -\beta\frac{S_{i}(t)I_{i}(t)}{n_{i}}\\
\frac{dE_{i}(t)}{dt} & = & \beta\frac{S_{i}(t)I_{i}(t)}{n_{i}}-\omega E_{i}(t)\\
\frac{dI_{i}(t)}{dt} & = & \omega E_{i}(t)-\nu I_{i}(t)\\
\frac{dR_{i}(t)}{dt} & = & \nu I_{i}(t)
\end{eqnarray*}


where $\omega=$1 and $\nu=1/6$. We set $h(s)$=0 for $I_{i}(s)<0.5$
horses. 

Note that a property ``runs out'' of infectiousness as all the horses
recover. This is why the SIO model above does not have an ``R''
compartment -- ``O'' stage properties simply stop being infectious
after a while.
\end{document}
